\renewcommand{\hwnum}{3}

\title{Third Homework Assignment}
\author{\todo{}}
\date{due: 23 February 2022}
\maketitle

This homework is worth $7$ points.
The expectations for the third and fourth homework assignments are as follows:
\begin{itemize}
    \item Homework is due at 23:59 on the date specified.  Late assignments will
        not be accepted.
    \item Homework should be typeset, preferably using LaTex.
    \item Homework should be submitted as a single PDF and submitted on
        Gradescope.  On Gradescope, be sure to select the pages for each
        question (or else it might not be graded ...).
    \item Homework can be submitted as a group of up to three students.
        Collaboration across groups is encouraged, but written sholutions for
        each group must be independently written.
    \item Answers should be in complete sentences, and should make sense without
        seeing the question.
    \item You will not plagiarize, nor will you share your written solutions
        with classmates.  (But, again, discussing the questions across groups is highly encouraged).
    \item If you use the LaTex template ...
        \begin{itemize}
            \item List collaborators at the start of each question using the \texttt{collab} command.
            \item Put your answers where the \texttt{todo} command currently is (and
                remove the \texttt{todo}, but not the word \texttt{Answer}).
        \end{itemize}
    \item If you do not use the LaTex template ...
        \begin{itemize}
            \item If you collaborate or discuss the problems with anyone, it is
                expected that you note this at the top of the question.
            \item Please start each question on a new page.
        \end{itemize}
\end{itemize}


%%%%%%%%%%%%%%%%%%%%%%%%%%%%%%%%%%%%%%%%%%%%%%%%%%%%%%
%%%%%%%%%%%%%%%%%%%%%%%%%%%%%%%%%%%%%%%%%%%%%%%%%%%%%%
\collab{n/a}
\nextprob{Getting to Know You (1 point)}

\begin{enumerate}

    \item Change the photo in D2L to be a headshot so that I can recognize who you are.
        \paragraph{Answer}
        \todo{After you are done, remove this ``TODO'' and write `Done'.}

    \item What do you want to do after you graduate? (If submitting as a group,
        state what each individual member of the group hopes to do).
        \paragraph{Answer}
        \todo{}

    \item You have been in this class for a couple of weeks now.  Why do you
        think this class is so important for computer science majors?
        \paragraph{Answer}
        \todo{}

    \item What is the ``muddiest point'' in this class so far?  (That is, what
        concept / proof / definition have you seen that you are the least
        comfortable with?)  If there are multiple ``muddy points,'' list them.
        \paragraph{Answer}
        \todo{}

    \item This HW should be submitted as a PDF.
        For extra credit, submit a typeset document. For extra extra credit,
        use LaTex.

        No response needed here. Just select any page in Gradescope
        for this one.

\end{enumerate}

%%%%%%%%%%%%%%%%%%%%%%%%%%%%%%%%%%%%%%%%%%%%%%%%%%%%%%
%%%%%%%%%%%%%%%%%%%%%%%%%%%%%%%%%%%%%%%%%%%%%%%%%%%%%%
\collab{\todo{}}
\nextprob{Review and Practice Questions (1 point)}

\begin{enumerate}
    \item Is the following statement true: If $p$ is a prime number, then $p$ is an odd
        integer. Justify.\label{part:thestmt}
        \paragraph{Answer}
        \todo{}

    \item Negate the statement in Part \ref{part:thestmt}. above.
        \paragraph{Answer}
        \todo{}

    \item Is the negation in the previous part true or false? Justify.
        \paragraph{Answer}
        \todo{}

    \item Give the contrapositive of the statement in Part \ref{part:thestmt}. above.
        \paragraph{Answer}
        \todo{}

    \item Is the contrapositive that you gave in the previous part true or
        false? Justify.
        \paragraph{Answer}
        \todo{}

    \item Your friend gave you the following proof that multiples of $4$ are
        even: \emph{Let $n=100$.  Since $n=4\cdot 25$, we have that $n$ is a multiple
        of $4$.  Also, $n=2\cdot 50$; therefore, by definition of even, $n$ is
        even.}  Does this proof work?  Why or why not?
        \paragraph{Answer}
        \todo{}

    \item What is the definition of an even number?
        \paragraph{Answer}
        \todo{}

    \item Give a counter-example to the following statement: for $a,b \in \Z$,
        $$a^2=b^2 \implies a=b.$$
        \paragraph{Answer}
        \todo{}

\end{enumerate}

%%%%%%%%%%%%%%%%%%%%%%%%%%%%%%%%%%%%%%%%%%%%%%%%%%%%%%
%%%%%%%%%%%%%%%%%%%%%%%%%%%%%%%%%%%%%%%%%%%%%%%%%%%%%%
\collab{\todo{}}
\nextprob{Divisibility (1 point)}

Suppose $n$ is an integer whose decimal representation ends in $5$.  Prove that
$5$ divides $n$. (Hint: recall that the decimal number $532$ can be written as
the sum $5\cdot 100 + 3 \cdot 10 + 2 \cdot 1$).

\paragraph{Answer}
\todo{}

%%%%%%%%%%%%%%%%%%%%%%%%%%%%%%%%%%%%%%%%%%%%%%%%%%%%%%
%%%%%%%%%%%%%%%%%%%%%%%%%%%%%%%%%%%%%%%%%%%%%%%%%%%%%%
\collab{\todo{}}
\nextprob{Quotient-Remainder Theorem (1 point)}

Let $n$ be an integer.  Prove that $n$ can be written in one of the three forms:
$$
    n=3q
    \text{ or }
    n=3q+1
    \text{ or }
    n=3q+2
$$
for some integer $q$. (Hint: Use the QRT).

\paragraph{Answer}
\todo{}

%%%%%%%%%%%%%%%%%%%%%%%%%%%%%%%%%%%%%%%%%%%%%%%%%%%%%%
%%%%%%%%%%%%%%%%%%%%%%%%%%%%%%%%%%%%%%%%%%%%%%%%%%%%%%
\collab{\todo{}}
\nextprob{Indirect Proof (1 point)}

Consider the following statement:

For every integer $n$, if $n^2$ is odd, then
$n$ is odd.

\begin{enumerate}
    \item Prove that this statement is correct.
        \paragraph{Answer}
        \todo{}

    \item Explain what proof technique you used to prove this (direct proof,
        proof by contradiction, proof by contrapositive, other).
        \paragraph{Answer}
        \todo{}

    \item What is the converse of the if/else statement, and is the converse true?
        \paragraph{Answer}
        \todo{}
\end{enumerate}
