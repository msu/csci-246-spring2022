\renewcommand{\hwnum}{4}
\setcounter{probnum}{0}

\title{Fourth Homework Assignment}
\date{due: 2 March 2022}
\maketitle

This homework is worth $7$ points.

The expectations for the third and fourth homework assignments are as follows:
\begin{itemize}
    \item Homework is due at 23:59 on the date specified.  Late assignments will
        not be accepted.
    \item Homework can be handwritten, but typeset is preferred (and typeset in
        LaTex is especially preferred).
    \item Homework should be submitted as a single PDF and submitted on
        Gradescope.  On Gradescope, be sure to select the pages for each
        question (or else it might not be graded ...).
    \item Homework can be submitted individually or in partners (pairs of two).
        Collaboration across groups is encouraged, but written sholutions for
        each group must be independently written.
    \item Answers should be in complete sentences, and should make sense without
        seeing the question.
    \item You will not plagiarize, nor will you share your written solutions
        with classmates.  (But, again, discussing the questions across groups is highly encouraged).
    \item If you use the LaTex template ...
        \begin{itemize}
            \item List collaborators at the start of each question using the \texttt{collab} command.
            \item Put your answers where the \texttt{todo} command currently is (and
                remove the \texttt{todo}, but not the word \texttt{Answer}).
        \end{itemize}
    \item If you do not use the LaTex template ...
        \begin{itemize}
            \item If you collaborate or discuss the problems with anyone, it is
                expected that you note this at the top of the question.
            \item Please start each question on a new page.
        \end{itemize}
\end{itemize}


%%%%%%%%%%%%%%%%%%%%%%%%%%%%%%%%%%%%%%%%%%%%%%%%%%%%%%
%%%%%%%%%%%%%%%%%%%%%%%%%%%%%%%%%%%%%%%%%%%%%%%%%%%%%%
\collab{\todo{}}
\nextprob{Review and Practice Questions (1 point)}

\begin{enumerate}
    \item Does $6$ divide $180$? Justify.
        \paragraph{Answer}
        \todo{}

    \item What is the biggest remainder possible when you divide a number by $8$?
        \paragraph{Answer}
        \todo{}

    \item Is the following statement true or false? \emph{If $n$ is an integer,
        then $n$ is a rational number}. (Briefly~justify). \label{part:newstmt}
        \paragraph{Answer}
        \todo{}

    \item What is the contrapositive of the statement in Part~\ref{part:newstmt}? Is
        it true? (Briefly~justify).
        \paragraph{Answer}
        \todo{}

    \item What is the converse of the statement in Part~\ref{part:newstmt}? Is
        it true? (Briefly justify).
        \paragraph{Answer}
        \todo{}

    \item If we were to (attempt to) prove the statement in
        Part~\ref{part:newstmt} by contradiction, what would the supposition~be?
        \paragraph{Answer}
        \todo{}

    \item Use the Fundamental Theorem of Arithmetic to write $378$ as a product
        of prime numbers.
        \paragraph{Answer}
        \todo{}

    \item Find $q,r$ from the QRT for $n=25$ and $d=4$.
        \paragraph{Answer}
        \todo{}

\end{enumerate}

%%%%%%%%%%%%%%%%%%%%%%%%%%%%%%%%%%%%%%%%%%%%%%%%%%%%%%
%%%%%%%%%%%%%%%%%%%%%%%%%%%%%%%%%%%%%%%%%%%%%%%%%%%%%%
\collab{\todo{}}
\nextprob{Sequences (1 point)}

\begin{enumerate}

    \item What is the $n$-th number in the following sequence:
        $$
            0, -\frac{1}{3}, \frac{4}{9}, -\frac{1}{3}, \frac{16}{81},
            -\frac{25}{243}, \frac{36}{729}, \ldots?
        $$
        (For reference,
        the zeroth number is $0$, the $1$st number is $-\frac{1}{3}$, and the $3$rd
        number is $\frac{4}{9}$).
        \paragraph{Answer}
        \todo{}

    \item Let $a_0=0$ and let $a_k=2a_0+k$, for $k\geq 1$.
        Compute $\sum_{k=0}^3 a_{2k}$.
        \paragraph{Answer}
        \todo{}

    \item Let $b_k=\frac{k}{k+1}$, for $k\in \N$.
        Compute  $\prod_{i=0}^{10} b_{k}$.
        \paragraph{Answer}
        \todo{}

    \item Consider the following set: $\{ 1/n\}_{n=1}^{\infty} \cup \{ 0 \}$.
        Is this set well-ordered. Why or why not?
        \paragraph{Answer}
        \todo{}



\end{enumerate}

%%%%%%%%%%%%%%%%%%%%%%%%%%%%%%%%%%%%%%%%%%%%%%%%%%%%%%
%%%%%%%%%%%%%%%%%%%%%%%%%%%%%%%%%%%%%%%%%%%%%%%%%%%%%%
\collab{\todo{}}
\nextprob{Proof by Induction (2 points)}

Induction is an important concept going forward in your computer science
curriculum.  Please try (or re-try) questions in the lectures and the textbook,
as two induction problems is not enough to master the concept.  Prove the
following two statements using induction.

\begin{enumerate}
    \item The number of edges in a complete graph on $n$ vertices is
        $\frac{n(n-1)}{2}$. % Section 5.3, Problem 31

        \paragraph{Answer}
        \todo{}

    \item Suppose that $c_0,c_1,c_2,\ldots$ is the sequence defined as follows:
        $c_0=2$, $c_1=2$,and, for every $k \geq 2$, $c_k:=c_{k-1}+3\cdot c_{k-2}$.
        Prove that $c_n$ is even for each integer $n \geq 0$.

        \paragraph{Answer}
        \todo{}

\end{enumerate}

%%%%%%%%%%%%%%%%%%%%%%%%%%%%%%%%%%%%%%%%%%%%%%%%%%%%%%
%%%%%%%%%%%%%%%%%%%%%%%%%%%%%%%%%%%%%%%%%%%%%%%%%%%%%%
\collab{\todo{}}
\nextprob{Recursion (2 points)}

(Section 5.6, Exercise 40). A set of blocks contains blocks of heights $1$, $2$,
and $4$ centimeters.  Imagine constructing towers by piling blocks of different
heights directly on top of one another. (A tower of height $6$ cm could be
obtained using six $1$-cm blocks; three $2$-cm blocks; one $2$-cm block on the
bottom with one $4$-cm block on top; one $4$-cm block on bottom with one $2$-cm
block on top, and so forth.) Let $t_n$  be the number of ways to construct a
tower of height $n$ cm using blocks from the set. (Assume an unlimited supply of
blocks of each size.) Find a recurrence relation for $t_1, t_2, t_3, \ldots $.
Justify your answer.

\paragraph{Answer}
\todo{}

%%%%%%%%%%%%%%%%%%%%%%%%%%%%%%%%%%%%%%%%%%%%%%%%%%%%%%
%%%%%%%%%%%%%%%%%%%%%%%%%%%%%%%%%%%%%%%%%%%%%%%%%%%%%%
\collab{\todo{}}
\nextprob{More Recurrence Relations (1 point)}

Find the closed form of the following recurrence relations:

\begin{enumerate}
    \item $T(n) = T(n-1) + 5$, with $T(0)=1$.
        \paragraph{Answer}
        \todo{}

    \item $T(n) = 2 T(n-1) + n$, with $T(0)=1$.
        \paragraph{Answer}
        \todo{}

    \item $T(n) = 2 T(n-1) + 3^n$, with $T(0)=1$.
        \paragraph{Answer}
        \todo{}

    \item (Bonus) What is the asymptotic form of the closed forms above?
        \paragraph{Answer}
        \todo{}
\end{enumerate}
