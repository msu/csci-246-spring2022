\section*{Lecture 11: Divisibility and Unique Factorization (Section 4.4)}

\begin{definition}[Divides]
    Let $n,d$ be integers.  We define: $n$ is \emph{divisible by} $d$
    if and only if
    $d \neq 0$ and there exists an $k \in \Z$ such
    that $n=kd$.  Moreover, this can be said in several different ways, that all
    mean the same thing:
    \begin{itemize}
        \item $n$ is \emph{divisible by} $d$
        \item $n$ is a \emph{multiple} of $d$
        \item $d$ is a \emph{factor} of $n$
        \item $d$ is a \emph{divisor} of $n$
        \item $d$ \emph{divides} $n$ (written $d \mid n$).
    \end{itemize}
\end{definition}

Let's practice!  We will complete some of the following problems
in class.  What we do not
complete, please do them on your own.  These will not be collected and graded,
but you are expected to know how to answer these questions. If you have any
questions, please reach out to an instructor or the TA.


\begin{enumerate}
    \item Write the definition of divides using an existential quantifier and
        the $\iff$ symbol.
       \practice
    \item What is the negation of the previous statement?
       \practice
    \item Is $5$ divisible by $3$?
       \practice
    \pagebreak
    \item Is $20$ divisible by $3$?
       \practice
    \item Is $0$ divisible by $3$?
       \practice
    \item Let $a,b \in \Z$. Is $3a+3b$ divisible by $3$?
       \practice
    \item Let $a \in \Z$. Is $15a$ divisible by $3$?
       \practice
\end{enumerate}


\pagebreak

\begin{theorem}[Divisibility is Transitive]
    For all integers $a,b,c$, if $a \mid c$ and $b \mid c$, then $a \mid c$.
\end{theorem}
\begin{proof}
    Let $a,b,c \in \Z$.
    \proofspace
    Thus, $a \mid c$.
\end{proof}

We have already seen one definition of prime numbers: An integer $n$ is prime if
and only if $n>1$ and the only positive integer factors of $n$ are $1$ and
itself.  We can use the notion
of divides to provide an alternative definition for prime.

\begin{definition}[Prime (version 2)]
    Let $n \in Z$.  We say that $n$ is \emph{prime} iff its only positive
    integer divisors are $1$ and itself.
\end{definition}

Let's practice!  We will complete some of the following problems
in class.  What we do not
complete, please do them on your own.  These will not be collected and graded,
but you are expected to know how to answer these questions. If you have any
questions, please reach out to an instructor or the TA.


\begin{enumerate}
    \item True or False? For all integers $a$ and $b$, if $a \mid b$ and $b \mid
        a$, then $a=b$.
        \practice
    \item What are the factors of $15$?
        \practice
    \item What are the prime factors of $15$?
        \practice
    \item What are the factors of $45$?
        \practice
    \item What are the prime factors of $45$?
        \practice
\end{enumerate}

\begin{theorem}[Fundamental Theorem of Arithmetic (Unique Factorization of Integers)]
    Let $n>1$ be an integer.  Then, there exists a positive integer $k$ and
    distinct prime numbers $p_1, p_2, \ldots, p_k$ and integers $e_1, e_2,
    \ldots, e_k$ such that:
    $$ n=p_1^{e_1}p_2^{e_2} \cdot \ldots \cdot p_k^{e_k},$$
    and all other expressions for $n$ as a product of prime numbers is
    equivalent to this expression by reordering the prime numbers.  If $p_1 <
    p_2 < \ldots < p_k$, we say $n$ is written \emph{in standard factored form}.
\end{theorem}

In other words, every integer greater than $1$ is either prime or can be written as the
product of prime numbers in a way that is unique (up to the order in which
the primes are written).
