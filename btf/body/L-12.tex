\section*{Lecture 12: Quotient-Remainder Theorem (Section 4.5)}

\begin{theorem}[Quotient-Remainder Theorem (QRT)]
    Let $n \in \Z$ and $d \in \Z_{+}$.  Then, there exists unique integers $q$
    and $r$ such that
    $$ n = dq + r, $$
    where $0 \leq r < d$.
\end{theorem}

As computer scientists, we are familiar with this theorem.  In fact, we know
these as integer division (``div'') and modular arithmetic (``mod''):
$$q = n \text{ div } d \text{, and } r= n\mod d. $$
Different programming languages this slightly differently syntax for mod and
div (e.g., integer
division is~$n//d$
in python, but~$n/d$ in C++).

Let's practice!  We will complete some of the following problems
in class.  What we do not
complete, please do them on your own.  These will not be collected and graded,
but you are expected to know how to answer these questions. If you have any
questions, please reach out to an instructor or the TA.


\begin{enumerate}
    \item Find $q,r$ from the QRT for $n=27$ and $d=5$.
        \practice
    \item Find $q,r$ from the QRT for $n=25$ and $d=5$.
        \practice
    \item Find $q,r$ from the QRT for $n=21$ and $d=5$.
        \practice
    \pagebreak
    \item Find $q,r$ from the QRT for $n=0$ and $d=5$.
        \practice
    \item Find $q,r$ from the QRT for $n=-8$ and $d=5$.
        \practice
    \item Find $r$ from the QRT if $n$, $d$, and $q$ are known.
        \practice
    \item Today is Wednesday, 16 February 2022.  What day of the week is
        Wednesday, 18 February 2023? (Note: 2022 is not a leap year).
        \practice
\end{enumerate}

Note that $d=2$ is a special case.  In fact, $n$ is even iff $n \mod
2 = 0$.  And, $n$ is odd iff $n \mod 2 = 1$.  The \emph{parity} of an integer
refers to whether the integer is even or odd.

\begin{theorem}[Divisibility is Transitive]
    A necessary and sufficient condition for an integer $n$ to be divisible by a
    positive integer $d$ is that $n \mod d = 0$.
\end{theorem}
\begin{proof}
    Let $n \in \Z$ and let $d \in \Z_{+}$.

    First, we will prove the forward direction: \emph{If $n$ is divisible by
    $d$, then $n \mod d = 0$.}
    \proofspace

    Next, we will prove the backward direction: \emph{If $n \mod d = 0$,
    then $n$ is divisible by $d$.}
    \proofspace

    Thus, we have shown that $n \mod d = 0$ is a necessary and sufficient
    condition for $n$ to be divisible by $d$.
\end{proof}
