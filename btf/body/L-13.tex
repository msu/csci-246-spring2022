\section*{Lecture 13: Indirect Proofs (\textsection 4.7--8)}

We have already seen many logic statements in this class.  And, we have started
to prove that some of these statements are true (or false).  Let $P$ and $Q$ be
logical statements such that
$$ P \implies Q.$$
Generally, to prove this using a \emph{direct proof} we start with $P$ and make
one deduction after another to arrive at a sequence of statements (that are true
assuming $P$ is true) until we arrive at $Q$.

One \emph{indirect proof} technique is a \emph{proof by contradiction}  (or,
reduction to absurdity), where
we:
\begin{enumerate}
    \item Assume, by contradiction, that $P$ and $\neg Q$ are true.
    \item Make one deduction after another until a contradiction (an absurdity) is reached.
    \item Conclude that $P \implies Q$.
\end{enumerate}

\begin{theorem}[Greatest Integer]
    There is no greatest integer.
\end{theorem}
\begin{proof}
    Suppose, by contradiction, that
    \proofspace
    Therefore, there is no greatest integer, as was to be shown.
\end{proof}

\begin{theorem}[Rational + Irrational]
    The sum of a rational number and an irrational number is irrational.
\end{theorem}
\begin{proof}
    We rewrite the theorem statement as:

    \proofspace
\end{proof}

\pagebreak
A second indirect proof technique is an \emph{argument by contrapositive}.
Here, instead of proving $P \implies Q$, we~prove:
$$ \neg Q \implies \neg P. $$
In other words ``If $Q$ is false, then $P$ is false.''
Let's try an example.  But, first, a lemma:

\begin{lemma}[Even or Odd]
    Every integer $n$ is either even or odd, but not both.
\end{lemma}
You should be able to prove this lemma, but we will not do that here.  We can
just assume that it is true.

\begin{theorem}[Even Squares]\label{thm:evensquare}
    For every integer $n$, if $n^2$ is even, then $n$ is even.
\end{theorem}
\begin{proof}
    Let $n\in \Z$.  We wish to show that: if $n^2$ is even, then $n$ is even.
    Equivalently, we can show the contrapositive:

    \proofspace
\end{proof}

We conclude with another theorem with a proof by contradiction.  Recall that a
number $q$ is called \emph{rational} if there exists integers $a,b$ such tat
$q=\frac{a}{b}$.  By the Fundamental Theorem of Arithmetic, we can write $a$
and~$b$ as a product of primes, and cancel any common prime numbers.  As a
result, we obtain integers $n$ and~$m$ with no common factors such that
$q=\frac{m}{n}$.  We commonly refer to this process as writing $q$ as a reduced~fraction.

\begin{theorem}[Irrationality of $\sqrt{2}$]
    $\sqrt{2}$ is irrational.
\end{theorem}

\begin{proof}
    Suppose, by contradiction, that $\sqrt{2}$ is rational.  Then, there exists
    two integers $m,n$ with no common factors such that
    $$ \sqrt{2}=\frac{m}{n}.$$

    By squaring both sides, we obtain $2=\frac{m^2}{n^2}$.
    By multiplying both sides by $n^2$, we obtain $2n^2=m^2$.
    Hence, by definition of even, $m^2$ is even.
    By \thmref{evensquare}, we know that $m$ is also even.

    By definition of even, there exists an integer $k$ such that $m=2k$.
    Substituting $m=2k$ into the equation~$2n^2=m^2$, we obtain
    $2n^2=(2k)^2=4k^2$. Dividing both sides by $2$, we obtain~$n^2=2k^2$, which
    means that $n^2$ is even.  Again using \thmref{evensquare}, we know that $n$
    is even.  However, this is a contradiction, as both~$n$ and $m$ are even
    (meaning that they both have a factor of $2$),
    but we began by assuming that $m,n$ had no common factors.
\end{proof}
