\section*{Lecture 14: Sequences (\textsection 5.1)}

\paragraph{First, a note on modular arithmetic:}

Last week, (among other topics), we discussed ``mod'' and ``div'' operators.
Perhaps the case for ``div'' is clear (divide and round down for positive
numbers).  But, where ``mod'' is used might not be as clear.  Let $n \in \Z$.
\begin{enumerate}
    \item $n \mod 2$ lets us know whether $n$ is a multiple of $2$ or not.
    \item If we have a repeated pattern that lasts $k$ units long and then
        repeats, $n \mod k$ tells us which type of unit the $n$-th unit is.
        For example, if today is Monday, what day of the week is it in $100$
        days? If the current time is 14:05 and I start a program that I estimate
        will take $52$ hours to complete, at what time will it complete on the
        day it does complete?
    \item If we want to write a natural number in binary, then we find the first digit
        by finding the largest integer $i$ for which $n_i:=n\text{ div } 2^i >0$.
        For example, consider $n=7$.\\
        $n\text{ div } 2^0=$\\
        $n\text{ div } 2^1=$\\
        $n\text{ div } 2^2=$\\
        $n\text{ div } 2^3=$\\
        Now that we have that, to compute the second digit, we take:
        $n_{i-1}:=(n-n_i2^i) \text{ div } 2^{i-1},$\\
        and, in general, we have (using \emph{summation notation}):
        $$
            n_j := \left(n- \sum_{k=j+1}^{i} n_k2^k \right) \text{ div } 2^{j}.
        $$
        \practice
        (See Page $272$ of the textbook for a different algorithm that uses repeated
        division by $2$).
    \item Writing numbers base $k$ for some other $k \in \Z_{+}$ works very
        similarly!
\end{enumerate}

\begin{definition}[Sequence]
    A \emph{sequence} is an ordered list.  Most often, we think of a sequence as
    a function from the natural numbers $\N:=\{ 0,1,2, \ldots\}$ (or a subset of
    them) to some set of interest.

    If $N \subseteq \N$ and $X$ is the set of interest, a sequence is a
    function $f \colon N \to X$.  In short, we often write $f_i:=f(i)$.
\end{definition}

Sometimes the sequence is explicitly given (e.g., $f_i=2^i$ are the ordered
powers of two).  Other times, we have a pattern (such as $1,-1/2, 1/3, -1/5,
\ldots$) and we need to find the function that defines it ($g_i=\frac{(-1)^{i+1}}{i}$, for
$i\in \Z_+$).  In the latter, a great technique to try is guess-and-check.

\pagebreak
Let's practice!  We will complete some of the following problems
in class.  What we do not
complete, please do them on your own.  These will not be collected and graded,
but you are expected to know how to answer these questions. If you have any
questions, please reach out to an instructor or the TA.


\begin{enumerate}
    \item Consider the sequence over $\N$ by: $a_k :=\frac{k}{k+1}$.
        What are the first five terms?
        \practice
    \item Let $d \in \N$.  Consider the sequence $b \colon \N \to \N$ defined
        by $b(k):= k \text {div } d$.  What is the range of $b$?
        \practice
    \item What is the following sequence: $c_k := (-1)^{k}$?
        \practice
    \item What is the following sequence: $d_k := (-i)^{k}$?
        \practice
\end{enumerate}

In addition to summations of numbers in a sequence (as we saw in the computation
of numbers base two),
we might want to take products of numbers.

\begin{definition}[Product Notation]
    Let $a_1, a_2, \ldots, a_n$ be a sequence of real numbers.
    The product of these $n$ numbers is denoted as follows:
    $$
        \prod_{i=1}^n:= a_1\cdot a_2 \cdot \ldots \cdot a_n.
    $$
    Note that this can also be defined recursively for $n > 1$ as follows:
    $$
        \prod_{i=1}^n := a_n \cdot \prod_{i=1}^{n-1}
    $$
\end{definition}

For example, the factorial ($n!$) can be written in product notation:
$$
    n! := \prod_{i=1}^n i = 1 \cdot 2 \cdot \ldots \cdot n.
$$
(And, for convenience, we define $0!:=1$).

Summations and products behave nicely, as we see in the following theorem:

\begin{theorem}[Properties of Summations and Products]
    Let $m \leq n$ be natural numbers and
    let $\{a_i\}_{i=m}^n$ and~$\{b_j\}_{j=m}^n$ be two sequences of real
    numbers, and let $c \in \R$.  Then:
    $$\sum_{k=m}^n a_k + c \cdot \sum_{k=m}^n b_k = \sum_{k=m}^n (a_k+ c\cdot b_k)$$
    $$\left( \prod_{k=m}^n a_k \right) \cdot \left(  \prod_{k=m}^n b_k \right)
        = \prod_{k=m}^n \left(a_k \cdot b_k\right)$$
\end{theorem}

Where do we see these in computer science? First, a definition:

\begin{definition}[Well-Ordered Set]
    A \emph{well-ordered set} is a set such that every subset has a smallest
    element. $\N$ is a well-ordered set, and the one we use often in CS!
\end{definition}

\paragraph{Application: Proofs of Termination}
In programming, we see sequences often.  Consider a variable that is changed in
a for loop.  Let $v_0$ be the initial value of that variable, and let $v_i$
denote the value of that variable after the $i$th time through the loop.
If the variable is valued in some well-ordered set (say, $\N$) and if
$v_i$ is always decreasing, then we know that the loop must terminate!
