\section*{Lecture 15: Proof by Induction (\textsection 5.2)}

This is a general purpose proof technique that works in a bottom-up fashion.
Knowing that a statement is true for a collection of instances, we argue that it
is also true for a new instance, which we then add to the collection.  Repeating
this step, we establish the statement for a countable collection.

We start with a familiar problem: summing the first $n$ positive integers.
%
\begin{lemma}
    For all $n \geq 0$, we have $\sum_{i=0}^n i = \frac{n(n+1)}{2}$.
\end{lemma}
\begin{proof}
    We proceed by induction.
    First, we note that $\sum_{i=0}^0 i = 0= \frac{0\cdot 1}{2}$.
    Let $n\geq 0$.
    Now, we assume inductively that
    \begin{eqnarray*}
        \sum_{i=0}^{n} i  &=&  \frac{n(n+1)}{2}.
    \end{eqnarray*}
    We now want to show that $\sum_{i=0}^{n+1} i  =  \frac{(n+1)(n+2)}{2}$.
    We have:
    \begin{align*}
        \sum_{i=0}^{n+1} i
            &=  \left(\sum_{i=0}^{n} i \right) + \left(\sum_{i=n+1}^{n+1} i \right)
            && \\%\text{by the properties of summations}\\
            &= \left(\sum_{i=0}^{n} i \right) + (n+1)
            && \\%\text{by expanding the sum}\\
            &= \frac{n(n+1)}{2} + (n+1)
            && \\%\text{by the inductive assumption}\\
            &= \frac{n(n+1)}{2} + \frac{2(n+1)}{2}
            && \\%\text{by multiplying by one}\\
            &= \frac{(n+1)(n+2)}{2}
            && \\%\text{by adding fractions,}
    \end{align*}
    as was to be shown. (Note: can you explain the steps above?)  Therefore, we conclude
    that, for all $n \geq 0$, we have $\sum_{i=0}^n i = \frac{n(n+1)}{2}$.
\end{proof}

To analyze why this proof is correct,
we let $p(k)$ be the statement that the claim is true for $n=k$.
Then, for $n \geq 0$, we assume $p(n) \implies p(n+1)$.
For $n=0$ we have $p(0) \wedge [p(0) \Rightarrow p(1)]$.
Hence, we get $p(1)$ by Modus Ponens.
We can see that this continues:
\begin{eqnarray*}
  p(0) \wedge [p(0)     \Rightarrow p(1)]  &\text{ hence }&  p(1); \\
  p(1) \wedge [p(1)     \Rightarrow p(2)]  &\text{ hence }&  p(2); \\
   \ldots &\ldots& \ldots \\
  p(n-1) \wedge [p(n-1) \Rightarrow p(n)]  &\text{ hence }&  p(n); \\
   \ldots &\ldots& \ldots
\end{eqnarray*}
Thus, $p(n_0)$ and $p(n) \Rightarrow p(n+1)$ for all $n \geq n_0$
implies $p(n)$ for all $n \geq n_0$.

We formalize the proof technique into the first, weak form of
the principle.
The vast majority of applications of Mathematical Induction
use this particular form.

\begin{definition}[Mathematical Induction (Weak Form)]
    If the statement $p(n_0)$ is true, and the
    statement $p(n) \Rightarrow p(n+1)$ is true for all $n \geq n_0$,
    then $p(n)$ is true for all integers $n \geq n_0$.
\end{definition}

To write a proof using the weak form of Mathematical Induction,
we thus take the following five steps:
it should have the following components:
\begin{enumerate}
    \item Introduce the proof.
    \item Base Case:            $p(n_0)$ is true.
    \item Inductive Hypothesis: Let $n \geq n_0$. Assume $p(n)$ is true.
    \item Inductive Step:       Prove $p(n+1)$.
    \item Inductive Conclusion: $p(n)$ for all $n \geq n_0$.
\end{enumerate}
Very often but not always, the inductive step is the most
difficult part of the proof.
In practice, we usually sketch the inductive proof,
only spelling out the portions that are not obvious.

If we can guess the closed form expression for a finite sum,
it is often easy to use induction to prove that it is correct,
if it is.

\begin{lemma}
    For all integers $n \geq 1$, we have $\sum_{i=1}^n 2^{i-1} = 2^n-1$.
\end{lemma}
\begin{proof}
    We prove the claim by the weak form of the
    Principle of Mathematical Induction.
    We observe that the equality holds when $n=1$
    because $\sum_{i=1}^1 2^{i-1} = 1 = 2^1-1$.
    Let $n \geq 1$.
    Assume inductively that the claim holds for $n$.
    \practice

    Thus, by the Principle of Mathematical Induction,
    $\sum_{i=1}^n 2^{i-1} = 2^n -1$ for all $n \geq 1$.
\end{proof}

Sometimes it is not enough to use the validity of $p(n-1)$ to derive $p(n)$.
Indeed, we have $p(n-2)$ available and $p(n-3)$ and so on.
Why not use them?
\begin{definition}{Mathematical Induction (Strong Form)}
    If the statement $p(n_0)$ is true and the statement
    $p(n_0) \wedge p(n_0+1) \wedge \cdots \wedge p(n) \Rightarrow p(n+1)$
    is true for all $n \geq n_0$,
    then $p(n)$ is true for all integers $n \geq n_0$.
\end{definition}
Notice that the strong form of the Principle of Mathematical Induction
implies the weak form.

\pagebreak
\begin{lemma}[Prime Factor Decomposition]
    Every integer $n \geq 2$ is the product of prime numbers.
\end{lemma}
\begin{proof}
    We prove the claim by the strong form of the
    Principle of Mathematical Induction.
    We know that $2$ is a prime number and thus also a product of prime numbers.
    Let $n \geq 2$.
    Suppose now that we know that every positive number up to (and including) $n$ is
    a product of prime numbers.
    \practice

    Therefore, by the strong form of the Principle of Mathematical Induction,
    every integer $n \geq 2$ is a product of prime numbers.
\end{proof}

\paragraph{Unique Prime Factor Decomposition}
We have used an even stronger statement before, namely that the
decomposition into prime factors is unique.
We can use the Reduction to Absurdity (proof by contradiction)
to prove uniqueness.
Suppose~$n$ is the smallest positive integer that has two
different decompositions.
Let $a \geq 2$ be the smallest prime factor in the two decompositions.
It does not belong to the other decomposition, else we could
cancel the two occurrences of $a$ and get a smaller integer
with two different decompositions.
Clearly, $n \bmod a = 0$.
Furthermore, for each
prime factor $p_i$ in the other decomposition of $n$,
$r_i := p_i \bmod a \neq 0$ .
We have
\begin{eqnarray*}
  n \bmod a  &=&  \left( \prod_i p_i \right) \bmod a             \\
             &=&  \left( \prod_i r_i \right) \bmod a.
\end{eqnarray*}
Since all the $r_i$ are smaller than $a$ and $a$ is a prime number,
the latter product can only be zero if one or the $r_i$ is zero.
But this contradicts that all the $b_i$ are prime numbers larger
than $a$.
We thus conclude that every integer larger than one has
a unique decomposition into prime factors.

\paragraph{Exercises}
Try to prove the following in your own time!
\begin{enumerate}
    \item For all $n \geq 0$, we have $\sum_{i=0}^n i = {n+1 \choose 2}$.
        Here, note that ${n+1 \choose 2} = \frac{n (n+1)}{2}$. (Weak Form).
    \item A tree with $n$ vertices has $n-1$ edges. (Strong Form).
\end{enumerate}

\paragraph{Summary.}
Mathematical Induction is a tool to prove that a property is true
for all positive integers.
We used Modus Ponens to prove the weak as well as the strong form
of the Principle of Mathematical Induction.
