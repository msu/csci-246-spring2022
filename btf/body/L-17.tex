\section*{Lecture 17: Solving Recurrence Relations (\textsection 5.7--5.9)}

Sometimes you have a hunch.  In this first
method, you might be able to \emph{guess} what the solution is, then prove it to
be true using ... you guessed it ... induction!  But, first, how do you come up
with an educated guess?

\paragraph{Method 1: Iteration} Here, we start with small values of $n$ and work
our way up until a pattern emerges. Let's try.

What is the closed form of the following recurrence relation?
(Note: this is the recurrence relation for the Towers of Hanoi!)
$$
    a_n = 2 a_{n-1} + 1, \text{ where } a_1=1.
$$
Now, let's work it out for small values of $n$ and see if we find a pattern.
\practice
% ... solution: a_n= 2^n-1.

Now that we've found that pattern, you should be able to use induction to prove
that it is correct!

\paragraph{Method 2: Plug and Chug} In this solution, you iteratively
\emph{plug} in the formula on the RHS then \emph{chug} away (usually using basic
algebra) to get the formula in a ``nice'' form.  Let's try.

$$
    b_n = b_{n-1} + 1, \text{ where } b_0=1.
$$
\practice
% ... solution: b_n = n-1
But, now, we have a guess for what the solution is ... to really prove it, we
need to use induction.

\pagebreak
\paragraph{Method 3: Recursion Tree} Now, we get a little visual.  Sometimes we
use the recursion tree to see if we can find a pattern.  Let's try.

$$
    c_n = c_{n/2} + 1, \text{ where } c_1=1.
$$
For now, let's pretend $n$ is a power of two.
\practice \vspace{1in}
% ... c_n = nlog(n)+n

Again, now, we have used the recursion tree to find
a guess for what the solution is ... to really prove it, we
need to use induction.

\paragraph{Asymptotic Forms}
In an earlier lecture, you learned about asymptotics (big-O, big-Omega, and
Theta notations).  What are the asmyptotic forms of the recurrrence relations
above?

\begin{itemize}
    \item $a_n$ is
    \item $b_n$ is
    \item $c_n$ is
\end{itemize}

We conclude with a few more examples.  Even if these are not covered in class,
try them at home for practice!

Let's practice!  We will complete some of the following problems
in class.  What we do not
complete, please do them on your own.  These will not be collected and graded,
but you are expected to know how to answer these questions. If you have any
questions, please reach out to an instructor or the TA.

{\bf What are the closed forms and asymptotic forms of the following recurrence
relations?}
\begin{enumerate}
    \item $d_n=d_{n-1}+2$, where $d_0=1$  % d_n=2^n-1
    \item $e_n = 1.1\cdot e_{n-1}$, where $e_0=1$. % e_n = (1.1)^n
    \item $f_n = r\cdot f_{n-1}$, where $f_0=a$, $r \geq 0$, and $a \in \R_+$. % f_n = a r^n
\end{enumerate}

\pagebreak

The following will not be on the quiz, but is good to know exists.

\paragraph{Master Method}
In a recurrence relation, we distinguish between
the \emph{homogeneous} part, the recursive terms,
and the \emph{inhomogeneous} part, the work that occurs.
The solution of depends on the relative size of the two,
exhibiting qualitatively different behavior if one dominates
the other or the two are in balance.
Recurrence relations that exhibit this three-regime behavior are so
common that it seems worthwhile to study this behavior in more detail.
We summarize the findings for how to obtain asymptotic forms.
\begin{theorem}{Master Theorem.}
    Let $a \geq 1$, $b > 1$, and $d > 0$ be real numbers,
    and let $f \colon \R \to \R$ be a function.
    Let~$T(n)$ be defined by the following recurrence relation:
    \begin{eqnarray*}
        T(n)  &=&  \left\{ \begin{array}{ccc}
            a T(\frac{n}{b}) + f(n)  &  \mbox{\rm if}  &   n > 1  \\
                    d                       &  \mbox{\rm if}  &   n = 1 .
        \end{array} \right.
    \end{eqnarray*}
    Then, we have the following:
    \begin{itemize}
        \item CASE 1: If $f(n)=O(n^{\log_b a - \epsilon})$ for some $\epsilon>0$,
            then $T(n) = \Theta (n^{\log_b a})$;
        \item CASE 2: If $f(n)=\Theta(n^{\log_b a})$, then $T(n) = \Theta (n^{\log_b a} \log n)$;
        \item CASE 3:  If $f(n)=\Omega(n^{\log_b a + \epsilon})$ for some $\epsilon>0$
            and if there exists $c>0$ such
            that~$af(n/b) \leq cf(n)$ for all large~$n$,
            then $T(n) = \Theta ( f(n) )$.
    \end{itemize}
\end{theorem}

Note: when $\frac{n}{b}$ is not an integer, we round using either the floor or
the ceiling function.

The recurrence relations for merge sort and for binary search satisfy this
pattern!  Which case do they fall into?
\practice
