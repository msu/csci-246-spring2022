\section*{Module 2 of CSCI 246 (Spring 2022)}

Important course links:
\begin{itemize}
    \item \href{https://drive.google.com/drive/folders/15M83uQne8Y-jddRmr5QYIur-tM_K77VQ?usp=sharing}{Google
    Drive}
    \item \href{https://github.com/msu/csci-246-spring2022}{Repo on GitHub}
\end{itemize}

\paragraph{Land Acknowledgement}

Living in Montana, we are on the ancestral lands of American Indians, including
the 12 tribal nations that call Montana home today: A’aninin (Gros Ventre),
Amskapi/Piikani (Blackfeet), Annishinabe (Chippewa/Ojibway), Annishinabe/M\'etis
(Little Shell Chippewa), Apsáalooke (Crow), Ktunaxa/Ksanka (Kootenai), Lakota,
Dakota (Sioux), Nakoda (Assiniboine), Ne-i-yah-wahk (Plains Cree), Qíispé (Pend
d’Oreille), Seli\v{s} (Salish), and Tsétsêhéstâhese/So’taahe (Northern Cheyenne).
We honor and respect these tribal nations as we live, work, learn, and play in
this state.

To learn more about Montana Indians, I suggest starting with the following
pamphlet:
\href{http://opi.mt.gov/Portals/182/Page%20Files/Indian%20Education/Indian%20Education%20101/essentialunderstandings.df}{Essential Understandings Regarding Montana Indians}


\paragraph{About Dr.~Fasy}
Dr.~Brittany Terese Fasy will be teaching three weeks of CSCI 246, starting
2/14/22.  So you know a little about her ...

\begin{itemize}
    \item Born in: Philadelphia, PA
    \item Descends from pirates
    \item BS from Saint Joseph's University in Computer Science and Mathematics
        (Philadelphia, PA)
    \item PhD from Duke University in Computer Science (Durham, NC)
    \item Visiting Scientist at IST Austria (lived in Vienna, Austria)
    \item Postdoc \#1: Carnegie Mellon University (Pittsburgh, PA)
    \item Postdoc \#2: Tulane University (New Orleans, LA)
    \item Fall 2015: Started at MSU
    \item 2021: Earned tenure at MSU
    \item Sports: lacrosse, skiing, yoga
    \item Hobby: cooking
\end{itemize}

\paragraph{Contacting Dr.~Fasy}
The best way to reach me is through email (brittany.fasy@montana.edu).  I have
office hours via Zoom Tuesdays and Thursdays 10:40-11:55, and can often be
caught right after class for quick questions.  See link in D2L or
email to schedule.
